%
% File acl-hlt2011.tex
%
% Contact: gdzhou@suda.edu.cn
%%
%% Based on the style files for ACL2008 by Joakim Nivre and Noah Smith
%% and that of ACL2010 by Jing-Shin Chang and Philipp Koehn


\documentclass[11pt]{article}
\usepackage{acl-hlt2011}
\usepackage{adjustbox}
\usepackage{amsmath}
\usepackage{bm}
\usepackage{graphicx}
\usepackage{latexsym}
\usepackage{multirow}
\usepackage{times}
\usepackage{url}
\usepackage{xcolor}

\DeclareMathOperator*{\argmax}{arg\,max}
\setlength\titlebox{6.5cm}    % Expanding the titlebox

\title{Predicting Rating Star Based on Sentiment Analysis of Yelp Review Text}

\author{Jiao Chen \\
  Computer Science and Engineering\\
  {\tt chenjiao@msu.edu} \\\And
  Xi Yin\\
  Computer Science and Engineering\\
  {\tt yinxi1@msu.edu} \\}

\date{}

\begin{document}
\maketitle
\begin{abstract}
  The Yelp challenge dataset includes information about local businesses, reviews and users in $10$ countries across $4$ countries.
  It can be used in various innovative ways to provide insights into business.
  In this project, we are interested in the sentiment analysis of the Yelp review texts.
  Particularly, we select the review texts and the corresponding rating stars as our dataset for training and testing.
  The sentiment analysis problem is transferred into star rating estimation, which can be treated as both a classification and a regression problem.
  We extract bag of words and bag of phrases as our feature and apply various techniques in classification and regression.
  Moreover, we apply the Stanford sentiment analysis tool to predict the sentiment value for each sentence in the review and compare it with our results.
\end{abstract}


\section{Introduction}
Large Scale Image Classification has been a very popular task in recent years.  It's a very important task in the field of computer vision for bridging the large semantic gap between an image-just a set of pixels  and  the object it presents. Researchers have build quite a few benchmark datasets for image classification such as MNIST, Label Me etc. Researchers also developed a lot of local descriptors, Bag Of Visual Words Model(BOVW), and different classification methods such as logistic regression, support vector machine etc. 

Just in past two or three years, deep learning began to bloom for many computer vision tasks, especially for image classification. To learn about thousands of objects from millions of images, we need a model with a great capacity. Convolutional neural network(CNN) provides us such an option. By utilizing multiple layers, nonlinear transformation etc. properties, CNN can effectively learn good feature representation and achieve state-of-the-art results, especially on large scale dataset such as ImageNet Dataset which contains over 15 million labeled labeled high-resolution images belonging to roughly 22,000 categories.  

In this project, we are provided with a modified version of ImageNet Dataset. This modified dataset consists of 1,262,106 images distributed over 164 classes. Some of the classes are directly from the ImageNet dataset, while others are generated by merging multiple classes in order to make it more challenging. Each image in the dataset is represented by a vector of 900 dimensions, and is assigned to one of 164 classes. We think it may be the raw count of  clustering results of SIFT features.  1,000,000 images were randomly chosen to form the training data, and the rest images as testing data. 

In this work, we tried different methods including logistic regression(LR), support vector machine(SVM) and neural network(NN), and finally average different models' results to form the final prediction result. 
\input{report_prior.tex}
%\input{report_tables.tex}
\section{Our Approach}
\label{sec:method}

\subsection{Feature extraction}
After we downloaded the Yelp reviews dataset, first we preprocess the dataset before extracting features. The original dataset is in json format, we use python \textit{json} module to extract review texts and their corresponding rating stars. All the words in review text are converted to lower case. Stop words are elimitated from the reviews by \textit{nltk.corpus.stopwords} and punctuations are also deleted by using regular expression. After preprocessing, we generate bag of words and bag of pharases feature vectors for each review text. The number of each unigram, bigram, trigram in a review text are counted and saved in the corresponding vectors.

\subsubsection{bag of words}
Bag of words (unigram) models are widely used in sentiment analysis of texts. Here we generate the unigram vocabulary from all the 1,600,000 review texts, resulting a dictionary with 414,197 unique words. However, most words in this dictionary are not related to sentiment and of low frequency. To improve the efficientcy of downstream analysis, we only keep the top 20,000 most frequenct words as candidate features. To make sure that sentiment related words will be in our features, we downloaded a sentiment vocabulary from previous work\cite{hu2004mining}, which includes 4,783 negative words, 2,006 positive words. So the total length of our unigram feature vector will be 26,789.

\subsubsection{bag of pharases}
In bag of pharases model (n-gram), we generate bigram and trigram features. Here for the size of our review texts, trigram feature vectors have already been very sparse, so we did not generate n-grams with n>3 features. For bigram features, the total number of unique bigrams from the rviews is 17,721,951. Noticing that a lot of bigrams occure only once in the reviews, we only keep bigrams occuring at least four times as our features. Thus, the bigram feature vector size is reduced to 2,858,126. For trigram features, the total number of unique trigrams is 68,126,384. Similar to bigram, we also only keep trigrams which occure at least four times as features. In this way, the trigram feature vector size is reduced to 2,145,801.

\subsection{Classification models}
Our goal is to predict the star rating (1, 2, 3, 4, 5) from the review text based on sentiment analysis. If we treat the star ratings as class labels, the problem can be treated as a mlti-class classification problem. We use Vowpal Wabbit (vw) (http://hunch.net/~vw/) machine learning tool to perform Logistic Regression, Support Vector Machine (SVM) and Neural Network classification methods on our datasets. VW is a fast online learning tool which was started at Yahoo! Research and continuing at Microsoft Research. Its default learning algorithm is a variant of online gradient descent.

In vw, for different classification models, the loss function can be written in the uniform way:\\
\begin{equation}
\sum_i(L(\textbf{x}_i,y_i,\textbf{w})+\lambda_1||\textbf{w}||_1+1/2\cdot\lambda_2||\textbf{w}||_2^2
\end{equation}
$\textbf{x}$ is the feature vector and $\textbf{w}$ is the parameter vector. $\lambda_1$ and $\lambda_2$ are the coefficients to specify the level of L1 and L2 regularization, respectively. Different classification models are identified by the loss function type. For example, the loss function for Logistic Regression is logistic loss: $ln(1+exp(-y\textbf{w}^T\textbf{x}))$, while loss function for SVM is hinge loss: $max(0,1-y\textbf{w}^T\textbf{x})$. Another important parameter for vw is the learning rate $\lambda$. Let $y_t$ be the ground truth class label for $t$th smaple, and $\hat{y_t}$ be the predicted class label. In online learning algorithm, the classification vector $\textbf{w}$ is updated in the way: If $y_t\neq\hat{y_t}$, then\\
\begin{equation}
\textbf{w}_{t+1}=\textbf{w}_t+\lambda y_t\textbf{x}_t
\end{equation}

Neural network is an interconnected group of nodes (neurons) which can compute values from inputs. A set of input neurons can be activated by the input data, the activations of these neurons are then passed to the other neurons. Repeat this process until finally an output neuron is activated. Here, we use neural network with 10 hidden layers for classifying the review samples.

\subsection{Regression model}
Considering that their is order between rating stars, that is 5$>$4$>$3$>$2$>$1. So rating 1 and 2 is more close than rating 1 and 5. We use linear regression to perform ordinal classification on our dataset.  In general, for each sample feature vector $\textbf{w}$, it is mapped to a real value $y(\textbf{x},\textbf{w})$:\\
\begin{equation}
y(\textbf{x},\textbf{w})=\textbf{w}^T\textbf{x}
\end{equation}
And $\textbf{w}$ is learned by minimizing the loss function for linear regression:\\
\begin{equation}
\textbf{w}^* = \underset{\textbf{w}}{\mathrm{arg min}}\frac{1}{2}\sum_{n=1}^N\left\{y_n-\textbf{w}^T\textbf{x}_n\right\}^2,
\end{equation}
The value $y(\textbf{x},\textbf{w})$ will be converted to an integer between 1 to 5 to represent for the star ratings.


































\input{report_evaluation.tex}
\input{report_discussion.tex}
\input{report_conclusion.tex}


\clearpage
{\small
\bibliographystyle{acl}
\bibliography{references}
}

%\begin{thebibliography}{}
%
%\bibitem[\protect\citename{Aho and Ullman}1972]{Aho:72}
%Alfred~V. Aho and Jeffrey~D. Ullman.
%\newblock 1972.
%\newblock {\em The Theory of Parsing, Translation and Compiling}, volume~1.
%\newblock Prentice-{Hall}, Englewood Cliffs, NJ.
%
%\bibitem[\protect\citename{{American Psychological Association}}1983]{APA:83}
%{American Psychological Association}.
%\newblock 1983.
%\newblock {\em Publications Manual}.
%\newblock American Psychological Association, Washington, DC.
%
%\bibitem[\protect\citename{{Association for Computing Machinery}}1983]{ACM:83}
%{Association for Computing Machinery}.
%\newblock 1983.
%\newblock {\em Computing Reviews}, 24(11):503--512.
%
%\bibitem[\protect\citename{Chandra \bgroup et al.\egroup }1981]{Chandra:81}
%Ashok~K. Chandra, Dexter~C. Kozen, and Larry~J. Stockmeyer.
%\newblock 1981.
%\newblock Alternation.
%\newblock {\em Journal of the Association for Computing Machinery},
%  28(1):114--133.
%
%\bibitem[\protect\citename{Gusfield}1997]{Gusfield:97}
%Dan Gusfield.
%\newblock 1997.
%\newblock {\em Algorithms on Strings, Trees and Sequences}.
%\newblock Cambridge University Press, Cambridge, UK.
%
%\end{thebibliography}

\end{document}
